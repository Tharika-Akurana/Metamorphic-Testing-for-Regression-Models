\section*{Abstract}

In order to carry out crucial functions including trajectory prediction, lane maintenance, steering control, and vehicle motion planning, autonomous driving systems are depending more and more on machine learning-based regression models. Even while these models have made great strides, it is still very difficult to guarantee their dependability and safety in actual driving situations. The test oracle problem, where it is challenging or impossible to discern the proper output for a given input, is one of the main challenges in validating such systems, especially when the model generates continuous-valued outputs. Therefore, traditional testing methods, which rely on predetermined scenarios and anticipated outcomes, are inadequate to ensure reliable system behavior in a variety of unexpected circumstances.

\vspace{0.5cm}
\setlength{\parindent}{0pt} 
By validating software behavior through metamorphic relations that specify expected interactions between many system executions under deliberately altered inputs, Metamorphic Testing (MT) has emerged as a promising solution to the Oracle problem. The lack of a universal, reusable, and scalable testing framework has prevented MT from being widely used in autonomous driving regression models, despite its successful application in other domains.

\vspace{0.5cm}
\setlength{\parindent}{0pt} 
Regression models used in autonomous driving systems are the focus of this project's design and implementation of a generic metamorphic testing platform. The suggested method allows for the systematic creation and assessment of test cases without the need for explicit test oracles by introducing a model-agnostic metamorphic testing template that abstracts common metamorphic features. Regression models for real-world autonomous driving and proof-of-concept tests are used to validate the framework, which is constructed as a modular Python-based system. A web-based interface is also created to improve usability and accessibility for practitioners and researchers.

The suggested platform seeks to enhance the robustness, dependability, and real-world preparedness of autonomous driving regression models by decreasing manual testing effort and expanding test coverage. This will help make the deployment of autonomous car technology safer.
\vspace{0.5cm}
\setlength{\parindent}{0pt} 