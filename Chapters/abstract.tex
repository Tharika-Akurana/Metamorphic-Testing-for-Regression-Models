\section*{Abstract}

In order to carry out crucial functions including trajectory prediction, lane maintenance, steering control, and vehicle motion planning, autonomous driving systems are depending more and more on machine learning based regression models. Even while these models have made great strides, it is still very difficult to guarantee their dependability and safety in actual driving situations due to the test oracle problem. A test oracle is defined as a mechanism capable of systematically verifying the correctness of a test result for any given input; however, a fundamental challenge arises when such an oracle either does not exist or is too expensive to be used. This problem presents a significant hurdle because it creates a verification difficulty, where it is impossible to clearly determine whether the continuous valued output of a program has passed or failed for a specific input. Furthermore, this leads to methodological restrictions, as the lack of an oracle limits the effectiveness of traditional test case selection strategies that rely on known outcomes. Consequently, while traditional testing is inadequate for ensuring reliable behavior in unexpected circumstances, alternative approaches like metamorphic testing seek to alleviate this problem by using metamorphic relations properties represented as relations among inputs and outputs of multiple executions to verify program correctness instead of relying on a single, predetermined outcome \cite{liu2014oracleMT}.

\vspace{0.5cm}
\setlength{\parindent}{0pt} 

By validating software behavior through metamorphic relations that specify expected interactions between many system executions under deliberately altered inputs, Metamorphic Testing (MT) has emerged as a promising solution to the oracle problem. The lack of a universal, reusable, and scalable testing framework has prevented MT from being widely used in autonomous driving regression models, despite its successful application in other domains.

\vspace{0.5cm}
\setlength{\parindent}{0pt} 
Regression models used in autonomous driving systems are the focus of this project’s design and implementation of a generic metamorphic testing platform. The suggested method allows for the systematic creation and assessment of test cases without the need for explicit test oracles by introducing a model agnostic metamorphic testing template that abstracts common metamorphic features. Regression models for real world autonomous driving and proof of concept tests are used to validate the framework, which is constructed as a modular Python based system. A web based interface is also created to improve usability and accessibility for practitioners and researchers.

The suggested platform seeks to enhance the robustness, dependability, and real world preparedness of autonomous driving regression models by decreasing manual testing effort and expanding test coverage. This will help make the deployment of autonomous car technology safer.
\vspace{0.5cm}
\setlength{\parindent}{0pt} 