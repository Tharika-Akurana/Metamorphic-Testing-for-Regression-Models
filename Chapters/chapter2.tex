\chapter{Literature Review}

\section{Background of the Autonomous Driving Systems and Validation}
Autonomous driving systems are safety critical cyber physical systems that integrate perception, decision making, and control components to enable vehicles to operate with limited or no human intervention \cite{sae2021visual} \cite{wef2025av}.
These systems continuously process data from multiple onboard sensors, including cameras, LiDAR, radar, and Global Positioning System (GPS) units, to perceive the surrounding environment and understand road geometry, traffic participants, and dynamic obstacles. Based on this perception, control actions such as steering, acceleration, braking, and lane keeping are generated to ensure safe vehicle operation.

\vspace{0.5cm}
\setlength{\parindent}{0pt} 
The operational environment of autonomous vehicles is inherently dynamic and uncertain. Factors such as varying road conditions, complex traffic interactions, weather changes, sensor noise, and unpredictable human behavior significantly increase system complexity \cite{thrun2006stanley} \cite{levinson2011towards}.
As a result, ensuring the reliability, robustness, and safety of autonomous driving software is a critical requirement throughout the development lifecycle. Failures or unexpected behaviors in these systems may lead to severe safety consequences, highlighting the importance of rigorous testing and validation methodologies \cite{iso26262}.

\vspace{0.5cm}
\setlength{\parindent}{0pt} 
Testing and validation play a central role in the development of autonomous driving systems. Conventional validation approaches typically rely on predefined datasets, simulation based testing, and scenario based evaluations using known traffic situations and standardized benchmarks \cite{dosovitskiy2017carla} \cite{nvidia2022simulation}.
While these methods enable controlled experimentation and functional verification, their effectiveness becomes limited when applied to machine learning based systems, particularly those producing continuous outputs and exhibiting non deterministic behavior.

\vspace{2cm}
\setlength{\parindent}{0pt} 
Many core components of autonomous driving systems, especially regression based machine learning models, generate continuous outputs such as steering angles, vehicle trajectories, velocity profiles, and control commands \cite{liu2024sgpr}. \cite{zhang2020mltesting}.
In real world driving scenarios, determining the correct output for a given input in advance is often infeasible. This leads to the well known test oracle problem, where the absence of explicit expected outputs prevents reliable automated validation \cite{liu2014oracleMT} \cite{barr2015oracleSurvey}.
Additionally, predefined datasets and scenario based testing fail to comprehensively capture rare edge cases and unseen driving conditions, making it difficult to guarantee robust real world performance.

\vspace{0.5cm}
\setlength{\parindent}{0pt} 
As autonomous driving systems increasingly rely on machine learning for perception and control, alternative validation techniques that do not depend on explicit expected outputs have gained increasing attention. This has motivated the exploration of metamorphic testing, which evaluates system correctness by verifying expected relationships between multiple executions rather than individual outputs \cite{zhou2019mtads} \cite{chen1998mt}.

\section{Regression Based Models in Autonomous Driving Systems}
For autonomous driving systems to operate safely and effectively, key functionalities such as trajectory prediction, lane centerline tracking, steering control, and motion planning must be performed reliably. Although deep neural network based methods have achieved strong performance in perception tasks, regression based models remain essential for control oriented and prediction based applications due to their interpretability, uncertainty estimation capabilities, and robustness when handling structured continuous data \cite{zhang2020mltesting} \cite{rasmussen2006gaussian}.

\vspace{0.5cm}
\setlength{\parindent}{0pt}
Among regression approaches, Gaussian Process Regression (GPR) has been widely adopted in autonomous driving for trajectory prediction and vehicle behavior modeling. GPR provides probabilistic predictions with uncertainty estimates, which are crucial for safety aware decision making in dynamic driving environments. However, traditional GPR suffers from high computational complexity, limiting its scalability in real time systems \cite{snelson2006sparse}.

\vspace{0.5cm}
\setlength{\parindent}{0pt}
To address this limitation, Sparse Gaussian Process Regression (SGPR) has been proposed as a scalable alternative that maintains prediction accuracy while enabling real time inference \cite{liu2024sgpr}.
SGPR based models have been successfully integrated into Model Predictive Control (MPC) frameworks for autonomous driving tasks such as lane keeping and overtaking maneuvers. Experimental studies demonstrate improved trajectory prediction accuracy and control stability across diverse traffic conditions \cite{zhou2019mtads} \cite{nguyen2020gpmpc}.

\begin{figure}[H]
\centering
\includegraphics[width=0.95\textwidth]{figures/Gaussian Process Trajectory Prediction.png}
\caption{Example of Gaussian Process based trajectory prediction in autonomous driving, showing predicted trajectories with uncertainty bounds for overtaking maneuvers \cite{liu2024sgpr}.}
\label{fig:Gaussian Process Trajectory Prediction}
\end{figure}

\setlength{\parindent}{0pt} 
An example of Gaussian Process based trajectory prediction in an autonomous driving setting is shown in Figure 2.1, which shows expected vehicle trajectories and related uncertainty bounds. Because they can give uncertainty estimates, which are essential for making safety aware decisions in dynamic driving conditions, these regression based models are very useful.

\vspace{0.5cm}
\setlength{\parindent}{0pt} 
Recent research presented at the IEEE Intelligent Vehicles Symposium further indicates that SGPR based trajectory prediction models can outperform deep learning approaches in terms of inference efficiency, interpretability, and adaptability, particularly when combined with geometric transformations such as translation and rotation equivariance \cite{liu2024sgpr} \cite{liu2023equivariant}.

\section{Limitations of Conventional Testing Approaches in Autonomous Driving}

Despite advances in modeling techniques, the testing and validation of autonomous driving systems remain a major challenge. Conventional testing approaches are often limited to predefined datasets, simulation based environments, and manually designed scenarios \cite{dosovitskiy2017carla} \cite{nvidia2022simulation}. 
While these methods provide controlled and repeatable evaluations, they struggle to represent the full complexity of real world driving conditions.

Autonomous vehicles operate in highly dynamic environments involving unpredictable agents, rare corner cases, and continuously evolving traffic situations \cite{thrun2006stanley} \cite{pei2017deepxplore}. 
As a result, models that perform well in controlled testing environments may exhibit unexpected or unsafe behavior when deployed in novel conditions. This limitation is particularly severe for regression based models with continuous outputs, where defining correct expected results for all possible inputs is impractical.

\vspace{0.5cm}
\setlength{\parindent}{0pt} 
Consequently, there is a growing need for systematic testing approaches that can evaluate model robustness beyond predefined datasets and scenarios, without requiring explicit output oracles \cite{liu2014oracleMT} \cite{barr2015oracleSurvey}.

\section{Metamorphic Testing: Previous Work}

Metamorphic testing (MT) was first introduced by Chen in 1998 as a software testing technique designed to address the oracle problem \cite{chen1998mt}.
Instead of relying on predefined expected outputs, MT verifies program correctness by examining relationships between multiple executions under systematically transformed inputs.

\vspace{0.5cm}
\setlength{\parindent}{0pt} 
These expected relationships, known as metamorphic relations (MRs), represent fundamental properties that the system under test should satisfy. If a metamorphic relation is violated, a fault is detected even when the correct output is unknown. This paradigm significantly reduces dependence on explicit test oracles and has proven effective for complex, data driven systems where traditional testing methods fail \cite{liu2014oracleMT}.

\begin{figure}[H]
\centering
\includegraphics[width=1\textwidth]{figures/Metamorphic Testing Workflow.png}
\caption{Conceptual overview of metamorphic testing, illustrating the generation of follow up test cases from source test cases using metamorphic relations to validate system behavior without an explicit test oracle \cite{qasource2020mt}.}
\label{fig:Metamorphic Testing Workflow
}
\end{figure}

\setlength{\parindent}{0pt} 
Instead of depending on explicit expected outputs, metamorphic testing uses preset metamorphic relations to generate follow up test cases from an initial source test case, as shown in Figure 2.2. The graphic illustrates how the oracle problem in complex systems, where accurate outputs are hard to predict ahead of time, is addressed by using linkages between many executions to check system behavior.

MT has since been successfully applied to numerical computation, search engines, cybersecurity, and machine learning systems, demonstrating its general applicability in oracle challenged environments \cite{zhang2020mltesting} \cite{zhou2018metamorphic}.

\section{Metamorphic Testing in Machine Learning and Autonomous Driving Systems}

In recent years, metamorphic testing has gained increasing attention in the validation of machine learning based systems, including autonomous driving applications \cite{zhou2019mtads} \cite{murphy2021testing}. Due to continuous outputs and complex system behaviors, MT is particularly suitable for testing Advanced Driver Assistance Systems (ADAS).

\vspace{0.5cm}
\setlength{\parindent}{0pt} 
Previous studies have applied geometric transformation based metamorphic relations, such as rotation, translation, and symmetry, to verify Lane Keeping Assist Systems (LKAS) under Euro NCAP driving scenarios \cite{mathworks2024lane} \cite{zhou2019mtads}.

\begin{figure}[H]
\centering
\includegraphics[width=1\textwidth]{figures/LKAS Functional Diagram.png}
\caption{ Diagram of a Lane Keeping Assist System (LKAS) in autonomous driving, illustrating lane detection, departure warning, and corrective steering \cite{mathworks2024lane}.}
\label{fig:LKAS Functional Diagram
}
\end{figure}
\setlength{\parindent}{0pt} 
A Lane Keeping Assist System (LKAS) usually includes of lane detection, vehicle position assessment, and corrective steering control components, as shown in Figure 2.3. Because geometric transformations and symmetry based metamorphic relations can be used to verify system consistency under various driving conditions, LKAS's modular structure makes it a good choice for metamorphic testing.

\vspace{0.5cm}
\setlength{\parindent}{0pt} 
Researchers have demonstrated that symmetry and rotation based MRs can reveal critical system faults that are often missed by conventional testing. By generating large numbers of follow up test cases from existing scenarios, MT enables systematic exploration of model behavior without relying on explicit expected outputs \cite{chen1998mt} \cite{tian2022metamorphic}.

\vspace{0.5cm}
\setlength{\parindent}{0pt} 
Additional metamorphic relations explored in the literature include translation invariance, trajectory preserving transformations, environmental condition variations, and sensor symmetry relations, which are particularly effective for testing regression based perception and control models \cite{zhou2019mtads} \cite{saeJ3016}.

\section{Gap Identification: Limitations of Existing Metamorphic Testing Approaches}
Despite its effectiveness, the practical adoption of metamorphic testing in autonomous driving remains limited. Most existing MT implementations rely on ad hoc test scripts, where metamorphic relations are manually defined and tightly coupled to specific models, simulators, or experimental setups \cite{zhou2019mtads} \cite{chen1998mt}.

\vspace{0.5cm}
\setlength{\parindent}{0pt} 
Although geometric transformation based MRs have proven successful, they are often implemented in a case specific and non reusable manner. This results in high manual effort, limited scalability, and poor integration into continuous development pipelines \cite{murphy2021testing}.

\setlength{\parindent}{0pt} 
\vspace{0.5cm}
Furthermore, current MT approaches typically require extensive domain knowledge and familiarity with simulation platforms, increasing development and maintenance overhead. These challenges are particularly pronounced for regression based autonomous driving models, where continuous outputs and complex interactions complicate MR design \cite{zhang2020mltesting}.

\section{Summary of Existing Systems and Research Gaps and way forward}
Existing autonomous driving systems widely employ regression based models such as GPR, SGPR, and learning based control models for trajectory prediction, lane keeping, and steering control \cite{liu2024sgpr} \cite{zhang2020mltesting} \cite{zhou2019mtads}.
While these models offer strong predictive capabilities and uncertainty estimation, their validation remains challenging due to the oracle problem and the limitations of conventional testing approaches \cite{liu2014oracleMT} \cite{barr2015oracleSurvey}.

\setlength{\parindent}{0pt} 
\vspace{0.5cm}
Although metamorphic testing has shown promise in addressing these challenges, current solutions lack abstraction, automation, and scalability. This clearly motivates the need for a generalized, reusable, and model agnostic metamorphic testing framework for autonomous driving regression models, which forms the basis of the proposed research.

