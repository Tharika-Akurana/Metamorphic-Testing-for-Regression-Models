\chapter{Literature Review}

\section{Regression-Based Models in Autonomous Driving Systems}

For autonomous driving systems to operate safely and effectively, key features like obstacle detection, decision-making, lane centerline management, and overtaking are essential. Because of their powerful representation learning capabilities, deep neural network-based methods have been increasingly used in recent years to tackle these issues. However, due to their interpretability, uncertainty estimates, and robust performance on structured, continuous data, regression-based models—especially Gaussian Process-based approaches—remain very important for autonomous driving applications (Chen et al., 2021). Among these, Sparse Gaussian Process Regression (SGPR) has drawn interest as a scalable substitute for conventional Gaussian Processes that preserves prediction accuracy while permitting real-time inference.

\begin{figure}[H]
\centering
\includegraphics[width=1\textwidth]{figures/Gaussian Process Trajectory Prediction.png}
\caption{Example of Gaussian Process-based trajectory prediction in autonomous driving, showing predicted trajectories with uncertainty bounds for overtaking maneuvers.}
\label{fig:Gaussian Process Trajectory Prediction}
\end{figure}

\setlength{\parindent}{0pt} 
An example of Gaussian Process-based trajectory prediction in an autonomous driving setting is shown in Figure 2.1, which shows expected vehicle trajectories and related uncertainty bounds. Because they can give uncertainty estimates, which are essential for making safety-aware decisions in dynamic driving conditions, these regression-based models are very useful.

\vspace{0.5cm}
\setlength{\parindent}{0pt} 
Model Predictive Control (MPC) frameworks for autonomous driving tasks, such as overtaking maneuvers, have effectively incorporated regression models like SGPR. For instance, SGPR is used by Gaussian Process-based MPC techniques to simulate vehicle dynamics and surrounding vehicle behavior in a variety of traffic scenarios. Experimental results show enhanced trajectory prediction accuracy and control stability across various driving circumstances, and these models are assessed using specified testing datasets (Frontiers in Neurorobotics, 2021). Similarly, SGPR-based trajectory prediction models can outperform deep learning approaches in terms of inference time, interpretability, and adaptability, especially when combined with geometric transformations like translation and rotation equivariance, according to recent IEEE Intelligent Vehicles Symposium research (Liu et al., 2024).

\section{Limitations of Conventional Testing Approaches in Autonomous Driving}

The testing and validation procedures used in many current autonomous driving projects, however, are frequently constrained to a small number of predetermined scenarios and datasets, despite their proven efficacy. It is challenging to fully capture the complexity and diversity of real-world driving conditions using conventional testing techniques because they include unforeseen impediments, uncommon edge cases, and extremely dynamic traffic interactions. As a result, even regression-based autonomous driving models that function effectively in carefully regulated experimental settings could behave erratically when used in novel or unusual circumstances. There is an increasing demand for either significantly larger and more varied test datasets or more generalized and systematic testing approaches that may assess model performance outside of predetermined situations in order to increase robustness and facilitate efficient fine-tuning of such models.

\vspace{2.5cm}
\section{Metamorphic Testing: Previous Work}
Metamorphic testing (MT) was first introduced by Chen (1998) as a systematic software testing technique designed to address the oracle problem, a common challenge in domains where it is difficult or impossible to determine the correct output for a given input in advance. Metamorphic testing examines the links between numerous executions of the same program under altered inputs to verify program correctness rather than depending on explicit predicted results. These anticipated interactions, referred to as metamorphic relations (MRs), reflect essential characteristics that the system being tested needs to meet. Even without a conventional test oracle, a problem is exposed if an MR is broken. MT greatly decreases reliance on explicit oracles by reorienting the attention from output correctness to relational consistency. It has been especially successful for complex, data-driven systems when traditional testing methods are inadequate (Chen, 1998).

\begin{figure}[H]
\centering
\includegraphics[width=1\textwidth]{figures/Metamorphic Testing Workflow.png}
\caption{Conceptual overview of metamorphic testing, illustrating the generation of follow-up test cases from source test cases using metamorphic relations to validate system behavior without an explicit test oracle \cite{qasource2020mt}.}
\label{fig:Metamorphic Testing Workflow
}
\end{figure}

\setlength{\parindent}{0pt} 
Instead of depending on explicit expected outputs, metamorphic testing uses preset metamorphic relations to generate follow-up test cases from an initial source test case, as shown in Figure 2.2. The graphic illustrates how the oracle problem in complex systems, where accurate outputs are hard to predict ahead of time, is addressed by using linkages between many executions to check system behavior.

\section{Metamorphic Testing in Machine Learning and Autonomous Driving Systems}
In order to overcome the shortcomings of conventional testing methods, metamorphic testing (MT) is being used more and more in machine learning-based systems and autonomous driving applications. Previous studies show that MT is especially useful for Advanced Driver Assistance Systems (ADAS), because complex settings and continuous system outputs make it challenging to define exact test oracles. For instance, Lane Keeping Assist Systems (LKAS) under Euro NCAP driving scenarios have been verified and validated using geometric-transformation-based metamorphic relations (MRs) in recent studies. 

\begin{figure}[H]
\centering
\includegraphics[width=1\textwidth]{figures/LKAS Functional Diagram.png}
\caption{ Diagram of a Lane Keeping Assist System (LKAS) in autonomous driving, illustrating lane detection, departure warning, and corrective steering \cite{mathworks2024lane}.}
\label{fig:LKAS Functional Diagram
}
\end{figure}
\setlength{\parindent}{0pt} 
A Lane Keeping Assist System (LKAS) usually includes of lane detection, vehicle position assessment, and corrective steering control components, as shown in Figure 2.3. Because geometric transformations and symmetry-based metamorphic relations can be used to verify system consistency under various driving conditions, LKAS's modular structure makes it a good choice for metamorphic testing.

\vspace{0.5cm}
\setlength{\parindent}{0pt} 
Researchers were able to successfully identify important system faults that were missed by standard testing alone by employing symmetry- and rotation-based MRs to methodically create a large number of follow-up test cases from existing situations. Without depending on explicit intended outputs, these methods use relationships between numerous executions—such as invariant vehicle behavior under scene rotations—to indirectly assess system correctness (Chen, 1998; Zhou et al., as cited in Chen et al). In addition to geometric transformations, other metamorphic relations explored in the literature include translation invariance, trajectory-preserving transformations, environmental condition variations, and sensor symmetry relations, all of which are well-suited for testing perception and regression-based control models in autonomous systems. However, despite their proven efficacy, the majority of current MT implementations in autonomous driving are still closely linked to particular case studies, simulation platforms, or manually created MRs, which restricts their scalability and reuse across many models and development pipelines.

\vspace{1cm}
\section{Gap Identification: Limitations of Existing Metamorphic Testing Approaches}
Although the fact that metamorphic testing (MT) has been demonstrated to be a useful tool for resolving the oracle problem and exposing hidden flaws in autonomous driving and machine learning systems, its practical use is still restricted because of a number of methodological and structural issues. The majority of current research uses ad hoc metamorphic test scripts, in which metamorphic relations (MRs) are created by hand and closely linked to particular models, simulation systems, or experimental setups (Chen et al., 1998; Chen et al., 2018). Geometric-transformation-based MRs like rotation and symmetry can effectively reveal critical system failures, as shown in earlier work on ADAS and Lane Keeping Assist Systems (LKAS). Nevertheless, these MRs are usually implemented in a case-specific and non-reusable manner, requiring significant manual effort for each new system or platform.

\setlength{\parindent}{0pt} 
\vspace{0.5cm}
Additionally, the lack of abstraction and automation restricts MT's scalability, even though it has been used in machine learning models and autonomous driving systems using transformations including rotation, translation, scaling, and symmetry-based relations (Chen et al., 2018; Zhou et al., 2020). As demonstrated by simulation-based ADAS testing environments, current MT implementations frequently need developers to have a thorough understanding of both the system being tested and the underlying simulation tools, which results in considerable development and maintenance overhead. This restriction is especially noticeable for regression-based autonomous driving models, where the design and implementation of MRs are made more difficult by continuous outputs, intricate environmental interactions, and dynamic scenarios.

\setlength{\parindent}{0pt} 
\vspace{0.5cm}
The LKAS and Euro NCAP case studies demonstrate that MT can find errors that are missed by traditional testing; however, wider adoption across models and development pipelines is hindered by the lack of a generic, user-friendly metamorphic testing platform. Thus, a systematic, reusable, and model-agnostic metamorphic testing approach specifically designed for autonomous driving regression models is clearly lacking. Reducing manual labor, increasing test coverage, and facilitating scalable and reliable validation all depend on closing this gap, which immediately motivates the suggested strategy and logically leads to the methodology described in the following section.
