\chapter{Literature Review}

\section{Background of the Autonomous Driving Systems and Validation}
With limited or no human dependence in vehicle controlling and operating autonomous driving systems use safety critical cyber physical systems which are doing decision making and controlling components \cite{sae2021visual} \cite{wef2025av}.
These systems continuously process data from multiple sensors in the vehicle, including cameras, LiDAR sensors, radar, and Global Positioning System (GPS) units, to understand road geometry, traffic participants, and dynamic obstacles overall the surrounding environment. Based on this idea, control actions like steering, acceleration, braking, and lane keeping are generated to make sure the safety of the vehicle operation. 

\vspace{0.5cm}
\setlength{\parindent}{0pt} 
The surrounding environment of autonomous driving vehicles is dynamic and uncertain. Factors like sudden variations of road conditions, complex traffic interactions and intersections, sudden weather changes, sensor noise and unpredictable human behavior largely increase system complexity and unpredictability \cite{thrun2006stanley} \cite{levinson2011towards}.
Hence, ensuring the reliability, robustness, and safety of autonomous driving software is an important requirement while developing the lifecycle. The proper and harder testing and validation of systems is required due to the high unpredictability and sudden variations of inputs \cite{iso26262}.

\vspace{0.5cm}
\setlength{\parindent}{0pt} 
Testing systems play a vital role in the development of ADS. Currently widely used validation techniques normally dependant on predefined datasets, simulation based testing, and scenario based evaluations using known traffic situations and standardized benchmarks \cite{dosovitskiy2017carla} \cite{nvidia2022simulation}.
While these techniques make it controlled for experimentation and functional verification, their efficiency becomes limited when applied to machine learning based systems, specially those generating continuous outputs and showing non deterministic behavior. 

\vspace{2cm}
\setlength{\parindent}{0pt} 
Most of the core components of ADS, especially regression based machine learning models, generate continuous outputs like steering angles, vehicle trajectories, velocity profiles, and control commands \cite{liu2024sgpr}. \cite{zhang2020mltesting}.
In real world driving scenarios, planning the correct output decision for a given input in advance is often impossible. This causes the well known test oracle problem, where the absence of exact expected outputs prevents reliable validation \cite{liu2014oracleMT} \cite{barr2015oracleSurvey}.
Furthermore, predefined datasets and scenario based testing fails to properly capture rare edge cases and unseen driving conditions, making it more difficult to validate sophisticated real world performance.

\vspace{0.5cm}
\setlength{\parindent}{0pt} 
As ADS heavily rely on machine learning for perception and control, alternative validation techniques that do not depend on exact expected outputs have attracted increasing attention. This has caused the exploration of metamorphic testing, which evaluates system correctness by verifying expected relationships between multiple executions rather than individual outputs \cite{zhou2019mtads} \cite{chen1998mt}.

\section{Regression Based Models in Autonomous Driving Systems}
Key functionalities like trajectory prediction, lane centerline tracking, steering control, and motion planning must be performed correctly to make sure the safety and efficiency and operational safety of ADS. Even though deep neural network based methods have accomplished strong evidence of efficiency in perception tasks, regression based models stays vital for control focused and prediction based applications because of their interpretability, uncertainty estimation capabilities, and robustness when handling continuous data \cite{zhang2020mltesting} \cite{rasmussen2006gaussian}.

\vspace{0.5cm}
\setlength{\parindent}{0pt}
Among regression approaches, Gaussian Process Regression (GPR) has been widely used in autonomous driving for trajectory prediction and vehicle behavior modeling. GPR provides probabilistic predictions with uncertainty estimates, which are essential for safety aware decision making in dynamic driving environments. However, conventional GPR is more vulnerable for its computational complexity, limiting its scalability in real time systems \cite{snelson2006sparse}.

\vspace{0.5cm}
\setlength{\parindent}{0pt}
To address this issue, Sparse Gaussian Process Regression (SGPR) has been introduced as a reliable solution that maintains accuracy while ensuring the acceptance of real time inference \cite{liu2024sgpr}.
SGPR based models have been successfully applied into Model Predictive Control (MPC) frameworks for autonomous driving tasks like lane keeping and overtaking maneuvers. Studies show improved trajectory prediction accuracy and control stability across various traffic conditions \cite{zhou2019mtads} \cite{nguyen2020gpmpc}.

\begin{figure}[H]
\centering
\includegraphics[width=0.95\textwidth]{figures/Gaussian Process Trajectory Prediction.png}
\caption{Example of Gaussian Process based trajectory prediction in autonomous driving, showing predicted trajectories with uncertainty bounds for overtaking maneuvers \cite{liu2024sgpr}.}
\label{fig:Gaussian Process Trajectory Prediction}
\end{figure}

\setlength{\parindent}{0pt} 
As an example for Gaussian Process based trajectory prediction in an autonomous driving setting can be shown in Figure 2.1, which indicates the expected vehicle trajectories and related uncertainty bounds. These regression based models are very useful since they can give uncertainty estimates, which are essential for making safety aware decisions in dynamic driving conditions. 

\vspace{0.5cm}
\setlength{\parindent}{0pt} 
Recent research presented at the IEEE Intelligent Vehicles Symposium further show that SGPR based trajectory prediction models can outperform deep learning approaches in terms of inference efficiency, interpretability, and adaptability, specially when combined with geometric transformations such as translation and rotation equivariance \cite{liu2024sgpr} \cite{liu2023equivariant}.

\section{Limitations of Conventional Testing Approaches in Autonomous Driving}

In spite of the advances in modeling technologies, the testing and validation of ADS remain a major challenge. Traditional testing techniques are normally limited to predefined datasets, simulation based environments, and manually designed driving scenarios \cite{dosovitskiy2017carla} \cite{nvidia2022simulation}. 
While these methods give proper evaluations, they struggle to weigh the full complexity of real world driving scenarios. 

Autonomous vehicles operate in highly dynamic environmental conditions involving unpredictable agents, rare edge cases, and continuously complicating traffic situations \cite{thrun2006stanley} \cite{pei2017deepxplore}. 
Therefore, models which perform well in controlled testing environments may show unexpected or unsafe behavior when they are exposed to untested novel conditions. This limitation is especially worse for regression based models with continuous outputs, where defining correct expected outputs for all possible inputs is impossible. 

\vspace{0.5cm}
\setlength{\parindent}{0pt} 
Henceforth, there is an increasing need for systematic testing techniques that can evaluate model robustness and efficiency beyond predefined datasets and scenarios, without needing exact output oracles \cite{liu2014oracleMT} \cite{barr2015oracleSurvey}.

\section{Metamorphic Testing: Previous Work}

Metamorphic testing (MT) was first introduced by Chen in 1998 as a software testing technique designated to address the oracle problem \cite{chen1998mt}.
Instead of relying on predefined expected results, MT verifies program accuracy by checking relationships between multiple executions under carefully transformed inputs. 

\vspace{0.5cm}
\setlength{\parindent}{0pt} 
These expected relationships, known to be metamorphic relations (MRs), represent basic properties that the system under test should achieve. If a metamorphic relation is violated, a fault is detected even if the correct output is unknown. This paradigm significantly reduces dependence on explicit test oracles and has proven effective for complex, data driven systems where traditional testing methods fail \cite{liu2014oracleMT}.

\begin{figure}[H]
\centering
\includegraphics[width=1\textwidth]{figures/Metamorphic Testing Workflow.png}
\caption{Conceptual overview of metamorphic testing, illustrating the generation of follow up test cases from source test cases using metamorphic relations to validate system behavior without an explicit test oracle \cite{qasource2020mt}.}
\label{fig:Metamorphic Testing Workflow
}
\end{figure}

\setlength{\parindent}{0pt} 
Instead of depending on explicit expected outputs, metamorphic testing uses prefound metamorphic relations to generate follow up test cases from an initial source test case, as shown in Figure 2.2. The diagram in the figure indicates how the oracle problem in complex systems, where correct outputs are hard to predict ahead of time, is addressed by using connections between many executions to check system behavior. 

MT has since been successfully applied to numerical computation, search engines, cybersecurity, and machine learning systems, showing its general applicability and reliability in oracle challenged environments \cite{zhang2020mltesting} \cite{zhou2018metamorphic}.

\section{Metamorphic Testing in Machine Learning and Autonomous Driving Systems}

In recent past few years, metamorphic testing has attracted a huge attention in the validation of machine learning based systems, including autonomous driving applications \cite{zhou2019mtads} \cite{murphy2021testing}. As a result of continuous outputs and complex system behaviors, MT is particularly suitable for testing Advanced Driver Assistance Systems (ADAS). 

\vspace{0.5cm}
\setlength{\parindent}{0pt} 
Previous studies have applied geometric transformation based metamorphic relations, such as rotation, translation, and symmetry, to verify Lane Keeping Assist Systems (LKAS) under Euro NCAP driving scenarios \cite{mathworks2024lane} \cite{zhou2019mtads}.

\begin{figure}[H]
\centering
\includegraphics[width=1\textwidth]{figures/LKAS Functional Diagram.png}
\caption{ Diagram of a Lane Keeping Assist System (LKAS) in autonomous driving, illustrating lane detection, departure warning, and corrective steering \cite{mathworks2024lane}.}
\label{fig:LKAS Functional Diagram
}
\end{figure}
\setlength{\parindent}{0pt} 
A Lane Keeping Assist System (LKAS) usually includes lane detection, vehicle position assessment, and accurate steering control components, as shown in Figure 2.3. Since geometric transformations and symmetry based metamorphic relations can be used to verify system consistency under various driving conditions, LKAS’s modular structure makes it a good case for metamorphic testing. 

\vspace{0.5cm}
\setlength{\parindent}{0pt} 
Researchers have shown that symmetry and rotation based MRs can reveal critical system faults that are often missed by current traditional testing. From generating huge numbers of follow up test cases from existing driving scenarios, MT enables systematic exploitation of the respected model behavior without depending on explicit or exact expected outputs \cite{chen1998mt} \cite{tian2022metamorphic}.

\vspace{0.5cm}
\setlength{\parindent}{0pt} 
Additional metamorphic relations explored in the literature include translation invariance, trajectory preserving transformations, environmental condition variations, and sensor symmetry relations, which are particularly effective for testing regression based perception and control models \cite{zhou2019mtads} \cite{saeJ3016}.

\section{Gap Identification: Limitations of Existing Metamorphic Testing Approaches}
In spite of its performance, the practical applications of metamorphic testing in autonomous driving remains limited. Most currently used MT applications rely on ad hoc test scripts, where metamorphic relations are manually defined and tightly coupled to specific models, simulators, or experimental setups \cite{zhou2019mtads} \cite{chen1998mt}.

\vspace{0.5cm}
\setlength{\parindent}{0pt} 
Although geometric transformation based MRs have proven successful, they are frequently implemented in a case specific and non reusable manner. This makes it high manual effort, limited scalability, and poor integration into continuous development pipelines \cite{murphy2021testing}.

\setlength{\parindent}{0pt} 
\vspace{0.5cm}
Furthermore, current MT approaches usually require extensive domain knowledge and familiarity with simulation platforms, increasing development and maintenance complexity for the users. These challenges are especially pronounced for regression based autonomous driving models, where continuous outputs and complex interactions complicate MR design \cite{zhang2020mltesting}.

\section{Summary of Existing Systems and Research Gaps and way forward}
Existing currently used autonomous driving systems widely use regression based models such as GPR, SGPR, and learning based control models for trajectory prediction, lane keeping, and steering contro \cite{liu2024sgpr} \cite{zhang2020mltesting} \cite{zhou2019mtads}.
While these models give strong prediction abilities and unpredictable estimation, their validation remains challenging because of the oracle problem and the limitations of traditional testing approaches \cite{liu2014oracleMT} \cite{barr2015oracleSurvey}.

\setlength{\parindent}{0pt} 
\vspace{0.5cm}
Although metamorphic testing has shown positivity in overcoming these challenges, current solutions lack abstraction, automation, and scalability. This surely motivates the need for a generalized, reusable, and model agnostic metamorphic testing framework for autonomous driving regression models, which makes the fundamental basis of the proposed research.

