\chapter{Conclusion}

The growing popularity of autonomous driving systems has brought attention to the urgent need for dependable and methodical testing techniques that can guarantee robust and safe model behavior in a variety of real-world scenarios. Due to their continuous outputs and lack of trustworthy test oracles, regression-based machine learning models—which are frequently employed in autonomous driving for tasks like trajectory prediction, steering control, and lane keeping—present particular testing issues. The complexity and diversity of real-world driving situations are frequently not well captured by conventional testing methods, which reduces confidence in model deployment.

\vspace{0.5cm}
\setlength{\parindent}{0pt} 
A generalized metamorphic testing platform created especially for regression models in autonomous driving systems is proposed in this study. Instead of depending on explicit expected outcomes, the suggested method uses metamorphic testing concepts to validate relational coherence across successive executions, so addressing the Oracle problem. Systematic, reusable, and scalable testing across a variety of regression models and autonomous driving applications is made possible by the development of a model-agnostic metamorphic testing framework.

\vspace{0.5cm}
\setlength{\parindent}{0pt} 
The manual effort needed to define and run metamorphic tests is greatly decreased by the use of a modular Python-based framework and an intuitive web-based interface. The suggested platform shows its efficacy in detecting any robustness and consistency problems that might be missed by traditional testing techniques through proof-of-concept demonstrations and validation using real-world autonomous driving regression models.

\vspace{0.5cm}
\setlength{\parindent}{0pt} 
All things considered, this work advances the robustness, dependability, and practicality of autonomous driving regression models. The suggested metamorphic testing platform facilitates the safer and more reliable deployment of autonomous driving technologies by bridging the gap between model development and systematic validation. It also offers a basis for further research and expansion in software testing for machine learning-based cyber-physical systems.
