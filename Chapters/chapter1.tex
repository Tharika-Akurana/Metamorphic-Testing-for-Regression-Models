\chapter{Introduction}
\setcounter{secnumdepth}{4}

With growing use in early autonomous vehicle platforms and commercial driver assistance systems, autonomous driving has become one of the most significant uses of artificial intelligence. The World Economic Forum states that although widespread full autonomy is still limited by technical, legal, and safety issues, vehicle automation technologies are already present on public roads, especially at Levels 2 and 2+, and are anticipated to gradually expand toward higher levels of autonomy over the next ten years \cite{wef2025av}. Current autonomous driving systems still have trouble operating dependably in a variety of real world scenarios, despite tremendous advancements in perception, prediction, and control models \cite{zhang2020mltesting}\cite{zhou2019mtads}.

\vspace{0.5cm}

\section{Background}
\begin{figure}[H]
\centering
\includegraphics[width=1\textwidth]{figures/SAE Driving Automation Levels.png}
\caption{SAE International Levels of Driving Automation (J3016 standard), illustrating the progression from no automation (Level 0) to full autonomy (Level 5) \cite{sae2021visual}.}
\label{fig:SAE Driving Automation Levels}
\end{figure}

In the SAE J3016 standard, there are six stages of automation of driving, starting with none. (Level 0) all the way to full automation (Level 5), as in Figure 1.1. Currently, the majority of commercially deployed systems act at Level 2 and 2+ and demand constant human supervision and just some automation. This is an illustration of the importance of sound testing. processes will be to ensure reliability and safety in the real implementation. 
\vspace{0.5cm}

\setlength{\parindent}{0pt} 
A major disadvantage that restricts the strength of au is lack of adequate and realistic testing. autonomous driving models. Autonomous systems operate in highly dynamic, com. complicated situations where listing and verification of all the situations in advance is impractical so de can be. ployment. In machine learning based systems which give continuous outputs such as tracks, turning angles or force, this has often been referred to as a problem. the oracle problem, in which it is difficult or even impossible to see the correct output. for a given test input. As has been stated above, the absence of credible test oracles is the bane of machine learning applications, e.g., autonomous driving and advanced driver assistant system (ADAS) \cite{barr2015oracle}, \cite{zhang2020mltesting}.

\begin{figure}[H]
\centering
\includegraphics[width=1\textwidth]{figures/Test Oracle Concept.png}
\caption{Illustration of the test oracle problem in software testing, highlighting the challenge of determining correct outputs without explicit expected results \cite{barr2015oracleSurvey}.}
\label{fig:Test Oracle Concept}
\end{figure}

\setlength{\parindent}{0pt} 
Figure 1.2 show the relationship between stimuli and observations with regard to software testing. Stimuli ($S$) represent all factors that can regulate the behaviour of the system under test (SUT), including explicit test inputs provided by the tester, environ mental conditions, configuration parameters, and sensor inputs. 
A subset of these stimuli, denoted as $I \subset S$, correlate with the explicit inputs directly applied to the system. Observations ($R$) stand in for all aspects of the system’s behaviour that can be measured after stimulation, including explicit outputs ($O \subset R$) as well as non functional properties like execution time, resource utilization, and internal state changes. Elements that do not belong to either the stimulus set or the observation set neither affect nor are affected by the execution of the SUT~\cite{barr2015oracleSurvey}.

Testing fundamentally contains applying stimuli to the system and observing the resulting behaviour. 
However, as demonstrated in the figure, this process alone does not guarantee the availability of a clear mechanism for determining whether the observed behaviour is correct. 
This challenge is formally captured by the concept of a test oracle, which is commonly defined as a predicate that determines whether a given program execution satisfies its considered specification. Formally, a test oracle can be expressed as a function

\begin{equation}
O : E \rightarrow \{\text{true}, \text{false}\},
\end{equation}

where $E$ dsymbolize the set of all possible executions of the system under test~\cite{barr2015oracleSurvey}.

The oracle problem occurs when such a function cannot be reliably defined, implemented, or evaluated for all executions.  
Barr \emph{et al.} stress the fact that this restriction is especially acute in the present day software systems, which tend to be non deterministic, highly complex, 3 and data driven and in which specifying the correct outputs is either impractical or impossible~\cite{barr2015oracleSurvey}.
This is particularly acute when it comes to machine learning autonomous driving systems, which generate continuous values which are reliant on intricate and changing environmental inputs. 
This therefore leads to the need to adopt alternative testing approaches like metamorphic testing, which minimize or do away with reliance on conventional test oracles~\cite{liu2014oracleMT,barr2015oracleSurvey,chen1998mt}.

\vspace{0.5cm}
\setlength{\parindent}{0pt} 
Chen first came up with Metamorphic Testing in 1998. Metamorphic Testing is an useful way to deal with the oracle problem. Metamorphic Testing checks if a system has the qualities it needs which are called Metamorphic Relations. These Metamorphic Relations are like rules that apply to runs of the system with related inputs rather than just looking at what the system should output. Even if we do not know what the correct output is we can still find errors if these rules are not followed. Many areas have used Metamorphic Testing. It has worked well. These areas include navigation software, search engines, cyber security, machine learning systems and autonomous driving applications. Metamorphic Testing has been very effective, in all these fields\cite{chen1998mt}, \cite{zhou2019mtads}. Geometric transformation based metamorphic relations have proven particularly successful in exposing discrepancies under perspective, spatial, or environmental alterations in the context of autonomous systems problem statement.

\section{Problem Statement}
Though metamorphic testing has its benefits, it remains difficult to apply the testing in actual practice. The existing methods often involve actions by the developers to construct metamorphic relations manually and execute certain test scripts with each model and scenario. Due to the inconsistent test design, This is an inadequate domain coverage method, which is time-consuming, and may contain error-prone, and even fail to capture important edge instances. Consequently, the metamorphic testing is not a readily available and a systematically integrated part of the autonomous driving regression model creation process yet.

\section{Objectives and Scope}

\subsection{Project Goal}
Creating and implementing a generalized metamorphic testing environment for autonomous driving regression models is the main objective of this project. In order to facilitate efficient testing of both recently created and current autonomous driving models, the suggested platform seeks to streamline the definition, implementation, and assessment of metamorphic relations.

\vspace{2cm}
\subsection{Objectives}
In accomplishing the above project goal the following objectives are established. Each objective is with transparent verification measures to measure completion success.

\subsubsection{Objective 1: Identify and formalize applicable metamorphic relations for autonomous driving regression models}
This goal aims at establishing significant metamorphic relationships (e.g. rotation, trans). real world driving properties are represented by relation, symmetry, and invariance.

\vspace{0.2cm}
\setlength{\parindent}{0pt} 
Verification metrics:
\begin{itemize}
    \item •	Definition of at least six metamorphic relations that can be applied to autonomous driving.
    \item •	Specification of the input output constraints of 100\% of defined relations.
\end{itemize}

\subsubsection{Objective 2: Design a generalized and model agnostic metamorphic testing template}
This goal would guarantee that the methodology of testing can be replicated in other regression models that do not need model specific test oracles.

\vspace{0.2cm}
\setlength{\parindent}{0pt} 
Verification metrics:
\begin{itemize}
    \item Effective initiation of the template of at least three regression models.
    \item No structural adjustment is needed in implementing the template on models.
\end{itemize}

\subsubsection{Objective 3: Validate the proposed framework using representative autonomous driving regression models}
This objective shows that the framework is really useful and works well in life situations. The framework is something that can be applied in a way and it is effective. This means the framework is good, for use and it gets the job done under realistic testing scenarios.

\vspace{0.2cm}
\setlength{\parindent}{0pt} 
Verification metrics:
\begin{itemize}
    \item Application of the framework to at least two autonomous driving regression models.
    \item Execution of at least five metamorphic relations per model.
    \item Identification of violations or robustness confirmation for each evaluated model.
\end{itemize}

\subsubsection{Objective 4: Develop a web based interface for improved accessibility and usability}
This objective makes sure that the framework can be used by people who work with it every day without needing to know a lot, about the framework. The framework is something that people should be able to use knowledge of metamorphic testing implementation.

\vspace{0.2cm}
\setlength{\parindent}{0pt} 
Verification metrics:
\begin{itemize}
    \item A functional web interface supporting model selection, metamorphic relation configuration, and result visualization.
    \item Successful end to end test execution through the interface with minimal user inter action.
\end{itemize}

\subsection{Scope of the Project}
The scope of this project is defined to ensure clarity and feasibility and is outlined as follows: 

\begin{itemize}
    \item This project is limited to the regression based models applicable in autonomous driving activities such as lane keeping, trajectory prediction and estimation of steering angle.
    \item Metamorphic testing is done to manage the oracle problem in machine learning based systems.
    \item The suggested framework is model-neutral and runs on the software testing level, it does not substitute the current simulation platforms or vehicle control systems. 
    \item Validation is done based on simulated driving conditions and software level testing as opposed to actual vehicle deployment. 
    \item Classification models, sensor hardware validation, perception pipelines and low level vehicle actuation control are also explicitly beyond the framework of this work.
\end{itemize}


